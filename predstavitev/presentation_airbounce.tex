%%%%%%%%%%%%%%%%%%%%%%%%%%%%%%%%%%%%%%%%%
% Beamer Presentation
% LaTeX Template
% Version 1.0 (10/11/12)
%
% This template has been downloaded from:
% http://www.LaTeXTemplates.com
%
% License:
% CC BY-NC-SA 3.0 (http://creativecommons.org/licenses/by-nc-sa/3.0/)
%
%%%%%%%%%%%%%%%%%%%%%%%%%%%%%%%%%%%%%%%%%

%----------------------------------------------------------------------------------------
%	PACKAGES AND THEMES
%----------------------------------------------------------------------------------------

\documentclass{beamer}

\mode<presentation> {

% The Beamer class comes with a number of default slide themes
% which change the colors and layouts of slides. Below this is a list
% of all the themes, uncomment each in turn to see what they look like.

%\usetheme{default}
%\usetheme{AnnArbor}
%\usetheme{Antibes}
%\usetheme{Bergen}
%\usetheme{Berkeley}
%\usetheme{Berlin}
%\usetheme{Boadilla}
%\usetheme{CambridgeUS}
%\usetheme{Copenhagen}
%\usetheme{Darmstadt}
%\usetheme{Dresden}
%\usetheme{Frankfurt}
%\usetheme{Goettingen}
%\usetheme{Hannover}
%\usetheme{Ilmenau}
%\usetheme{JuanLesPins}
%\usetheme{Luebeck}
%\usetheme{Madrid}
%\usetheme{Malmoe}
%\usetheme{Marburg}
%\usetheme{Montpellier}
%\usetheme{PaloAlto}
%\usetheme{Pittsburgh}
%\usetheme{Rochester}
%\usetheme{Singapore}
%\usetheme{Szeged}
\usetheme{Warsaw}

%\usecolortheme{albatross}
%\usecolortheme{beaver}
%\usecolortheme{beetle}
%\usecolortheme{crane}
%\usecolortheme{dolphin}
%\usecolortheme{dove}
%\usecolortheme{fly}
%\usecolortheme{lily}
%\usecolortheme{orchid}
%\usecolortheme{rose}
%\usecolortheme{seagull}
\usecolortheme{seahorse}
%\usecolortheme{whale}
%\usecolortheme{wolverine}

\usefonttheme{professionalfonts} 
\setbeamertemplate{caption}[numbered]
%\usetheme{}

\addtobeamertemplate{navigation symbols}{}{%
    \usebeamerfont{footline}%
    \usebeamercolor[fg]{footline}%
    \hspace{1em}%
    \insertframenumber/\inserttotalframenumber
}

\usepackage[utf8]{inputenc}
\usepackage[english]{babel}
\usepackage{physics}
\usepackage{amsmath}
\usepackage{bm}
\usepackage{amsfonts}
\usepackage{amssymb}
\usepackage{caption}
\usepackage{subcaption}
\usepackage{float}

%\setbeamertemplate{footline} % To remove the footer line in all slides uncomment this line
%\setbeamertemplate{footline}[page number] % To replace the footer line in all slides with a simple slide count uncomment this line

%\setbeamertemplate{navigation symbols}{} % To remove the navigation symbols from the bottom of all slides uncomment this line
}

\usepackage{graphicx} % Allows including images
\usepackage{booktabs} % Allows the use of \toprule, \midrule and \bottomrule in tables

%----------------------------------------------------------------------------------------
%	TITLE PAGE
%----------------------------------------------------------------------------------------

\title[2. Airbounce]{Problem no.2 - Airbounce}
\subtitle{IPT 2022}
\author[University of Ljubljana]{Team Slovenia \\ Presenter: Rok Grgi\v c Me\v sko}
\date{\vspace{-5ex}}

\begin{document}

\begin{frame}

\begin{figure}[H]
	\centering
	  \includegraphics[width=\textwidth]{naslovnica_ipt.png}
\end{figure}

\titlepage % Print the title page as the first slide

\vspace{-13mm}
\begin{figure}[H]
	\flushleft
	  \includegraphics[width=6cm]{fmf_logo.png}
\end{figure}

\end{frame}


%----------------------------------------------------------------------------------------
%	PRESENTATION SLIDES
%----------------------------------------------------------------------------------------

%------------------------------------------------


\begin{frame}

\begin{block}{Official problem statement}
When a Frisbee is thrown in a certain way it can be made to bounce in mid-air. Study the physics of this phenomenon.
\end{block}
\includegraphics[width=\textwidth]{primer_meta.png}

\end{frame}

%------------------------------------------------

\begin{frame}

\begin{block}{Ideas and hypotheses}
\begin{itemize}
\item Normal component of Frisbee velocity will decrease faster because of its shape.

\item Frisbee will appear to bounce in mid-air.
\end{itemize}
\end{block}

\end{frame}

%------------------------------------------------

\begin{frame}

\frametitle{Theoretical description}

\begin{block}{}
\begin{itemize}
\item Frisbee in the original video is stable. Angle to the ground is constant.

\item Assumptions:

\begin{itemize}
\item Frisbee keeps constant angle to the ground during the whole flight because of gyroscopic stability.

\item Frisbee travels in a straight line. (no Magnus effect...)
\end{itemize}
\end{itemize}
\end{block}

\end{frame}

%------------------------------------------------

\begin{frame}

\frametitle{Theoretical description}

\begin{Large}
Axis graphs:
\end{Large}

\begin{figure}[H]
	\centering
	\begin{minipage}{.5\textwidth}
	  \centering
	  \includegraphics[width=\textwidth]{graf_osi.png}
	  \captionof{figure}{\\ Ground coordinate system: N}
	\end{minipage}%
	\begin{minipage}{.5\textwidth}
	  \centering
	  \includegraphics[width=\textwidth]{osi_frisbeeja.png}
	  \captionof{figure}{\\ Coordinate system of Frisbee: D}
	\end{minipage}
\end{figure}

$ \theta = $ angle to the ground \qquad \qquad $ \alpha = $ angle of attack

\end{frame}

%------------------------------------------------

\begin{frame}

\frametitle{Theoretical description}

\begin{block}{Lift and drag force}
\begin{equation}
L = \dfrac{1}{2} A \rho C_L v^2 \qquad D = \dfrac{1}{2} A \rho C_D v^2
\end{equation}
\end{block}

\begin{block}{Lift and drag coefficient depending on angle of attack \cite{clanek}}
\begin{equation}
C_L = C_{L0} + C_{L \alpha} \alpha \qquad C_D = C_{D0} + C_{D \alpha} \alpha^2
\end{equation}
\end{block}

\end{frame}

%------------------------------------------------

\begin{frame}

\frametitle{Theoretical description}

\begin{block}{Cutoff}
\begin{itemize}
\item $C_D$ cutoff; when $C_D = 1.1$ (drag coefficient of a disc perpendicular to velocity)

\item $C_L$ cutoff; at stall angle = $25^{\circ}$
\end{itemize}
\end{block}

\begin{figure}[H]
	\centering	  \includegraphics[width=0.6\textwidth]{lift_drag_primer.png}
	  \caption{$C_{L0} = 0.188, C_{L \alpha}= 2.37, C_{D0} = 0.15, C_{D \alpha} = 1.24$ \cite{clanek}}
\end{figure}

\end{frame}

%------------------------------------------------

\begin{frame}

\frametitle{Theoretical description: Forces}

\begin{columns}[onlytextwidth]

\column{0.5 \textwidth}

\vspace{-10mm}
\begin{figure}[H]
	\centering
	  \includegraphics[width=\textwidth]{osi_frisbeeja.png}
	  \caption{Coordinate system of Frisbee: D}
\end{figure}

\vspace{-10mm}
\begin{gather*}
K = \dfrac{A \rho}{2 m} \qquad \tan \alpha = \dfrac{-v_2}{v_1}\\
v = \sqrt{v_1^2 + v_2^2}
\end{gather*}

\column{0.5 \textwidth}

\vspace{-10mm}

\begin{gather}
\begin{split}
\bm L = m K C_L v^2 \mqty(\sin \alpha \\ \cos \alpha)_D\\
\bm L = m K  C_L v \mqty(-v_2 \\ v_1\\)_D
\end{split}
\end{gather}

\begin{gather}
\begin{split}
\bm D = m K C_D v^2 \mqty(-\cos \alpha \\ \sin \alpha)_D\\
\bm D = m K C_D v \mqty(-v_1 \\ -v_2\\)_D
\end{split}
\end{gather}

\begin{gather}
\begin{split}
\bm F_{\bm g} = - m g  \mqty(\sin \theta \\ \cos \theta)_D
\end{split}
\end{gather}

\end{columns}

\end{frame}

%------------------------------------------------

\begin{frame}

\frametitle{Theoretical description}

\begin{gather}
m \bm a = \bm L + \bm D + \bm F_{\bm g}\\
\mqty( \dot{v_1} \\ \dot{v_2}\\)_D = K  C_L v \mqty(-v_2 \\ v_1\\)_D + K C_D v \mqty(-v_1 \\ -v_2\\)_D - g  \mqty(\sin \theta \\ \cos \theta)_D\\
%a_1 = -K (C_{L0} + C_{L \alpha} \alpha) v v_2 - K (C_{D0} + C_{D \alpha} \alpha^2) v v_1 - g \sin \theta\\
%a_2 = K (C_{L0} + C_{L \alpha} \alpha) v v_1 - K (C_{D0} + C_{D \alpha} \alpha^2) v v_2 - g \cos\theta
\mqty( \dot{d_1} \\ \dot{d_2}\\)_D = \mqty( v_1 \\v_2\\)_D
\end{gather}
Solve for: $d_1, d_2, v_1, v_2 $ and rotate to ground coordinate system N.
\begin{gather}
R = \mqty(\cos \theta & -\sin \theta \\ \sin \theta & \cos \theta)\\
\mqty(x \\ y)_N = R \mqty(d_1 \\ d_2)_D \qquad \mqty(v_x \\ v_y)_N = R \mqty(v_1 \\ v_2)_D
\end{gather}
\end{frame}

%------------------------------------------------

\begin{frame}

\frametitle{Experiment}

\begin{itemize}
\item Video analysis of a throw.
\item Problems:
\begin{itemize}
\item Frisbee is not stable as in the original video.
\item ?Parallax? error. Throw is not perpendicular to the camera.
\end{itemize}
\end{itemize}

\begin{figure}[H]
	\centering
	  \includegraphics[width=\textwidth]{primer_meta.png}
	  \caption{Example of a throw.}
\end{figure}

\end{frame}

%------------------------------------------------

\begin{frame}

\frametitle{Parallax error correction}

\(dx\) is measured, \(dx'\) is correct

\begin{gather}
k = \dfrac{\mathrm{final \; frisbee \; size}}{\mathrm{initial \;  frisbee  \; size}} \qquad l = \mathrm{lenght \; of \; a \; throw}\\
dx =  \mu(x) dx' = \left( \dfrac{k - 1}{l} x + 1 \right) dx' \\
x = \int_{0}^{x} \mu(x) \,dx = \dfrac{l}{k - 1} \left[ \ln((k - 1) x + l) - \ln l \right]
\end{gather}

\end{frame}

%------------------------------------------------

\begin{frame}

\frametitle{References}

\setbeamertemplate{bibliography item}[text]

\begin{thebibliography}{9}

\bibitem{clanek}
M. Hubbard, S. A. Hummel. \textit{Simulation of Frisbee Flight}. (2000). \url{https://www.researchgate.net/publication/253842372_Simulation_of_Frisbee_Flight}

\end{thebibliography}

\end{frame}

%------------------------------------------------

\end{document} 